\subsection{Biblioteka standardowa}

Na razie jet w postaci dwóch plików .uml: class.uml oraz usecase.uml definiujące prototypy dla podstawowych obiektów diagramu klas oraz diagramu przypadków użycia. Można je traktować jako gotowe zestawy obiektów najczęsciej wykorzystywanych elementów tychże diagramów.
W class.uml znajdują się:
\begin{itemize}
\item prototype base class \\
Definicja klasy. Możliwe klucze to: \textbf{name}(nazwa klasy),\textbf{stereotype}(nadanie stereotypu). 
\item prototype base relation \\
Definicja relacji. Możliwe klucze to: \textbf{name}, \textbf{arrow}(zwrot strzałki), \textbf{direction}(umiejscowienie strzałki, przy czym może mieć takie wartosci jak \textbf{none}, \textbf{source}, \textbf{target}, \textbf{both}), \textbf{source-count}/\textbf{target-count}(krotnosci zgodne z UML obiektu docelowego oraz źródłowego), \textbf{source-role}/\textbf{target-role}(role zgodnie z UML obiektu docelowego oraz źródłowego). Wymagane klucze to: \textbf{source-object}/\textbf{target-object}(nazwy obiektu docelowego oraz źródłowego potrzebne do pokazania relacji).
\item prototype base note \\
Definicja notatki. Możliwy klucz to \textbf{text} oznaczajacy tresć wewnątrz notatki. 
\item prototype relation generalization \\
Definicja relacji generalizacji zgodnie z UML. Oprócz tego że jej prototypem jest relacja, czyli ma skopiowane wszystkie jej wartosci kluczy. Dodatkowo ma przypisane wartosci(można je oczywiscie zmienić w pliku)dla kluczy: \textbf{arrow} oraz \textbf{direction}.
\item prototype relation aggregation \\
Definicja relacji agregacji zgodnie z UML. Podobnie jak generalizacja ma okreslone wartosci: \textbf{arrow} oraz \textbf{direction}.
\item prototype relation composition \\
Definicja relacji kompozycji zgodnie z UML. Podobnie jak generalizacja czy agregacja ma okreslone wartosci: \textbf{arrow} oraz \textbf{direction}.
\item prototype relation association \\
Definicja relacji asocjacjii zgodnie z UML. Podobnie jak generalizacja, agregacja czy kompozycja ma okreslone wartosci: \textbf{arrow} oraz \textbf{direction}.

\end{itemize}

W usecase.uml znajdują się:
\begin{itemize}
\item prototype base actor \\
Definicja aktora zgodnie z UML. Tylko klucz \textbf{name} jest możliwy do ustawienia.
\item prototype base usecase \\
Definicja przypadku użycia zgodnie z UML. Tylko klucz \textbf{name} jest możliwy do ustawienia.

\end{itemize}